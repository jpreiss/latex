% Bare minimum packages for math.
% No bibliography, theorem numbers, etc. to avoid conflict with style files.

\usepackage{amsmath}
\usepackage{amssymb}
\usepackage{amsthm}
\usepackage{mathrsfs}
\usepackage{mathtools}
\usepackage{nicefrac}
\usepackage{siunitx}

% Make the penalty for breaking inline math across lines much higher.
% Override in particular equations by inserting \allowbreak.
\binoppenalty=9999
\relpenalty=9999

% Black magic from StackExchange.
% Defines mathcal macros such that \cA = \mathcal{A}, and so on.
\def\mydefc#1{\expandafter\def\csname c#1\endcsname{\mathcal{#1}}}
\def\mydefallc#1{\ifx#1\mydefallc\else\mydefc#1\expandafter\mydefallc\fi}
\mydefallc ABCDEFGHIJKLMNOPQRSTUVWXYZ\mydefallc

% Sets of numbers.
\newcommand{\N}{\mathbb{N}}
\newcommand{\Z}{\mathbb{Z}}
\newcommand{\Q}{\mathbb{Q}}
\newcommand{\R}{\mathbb{R}}
\newcommand{\C}{\mathbb{C}}

% Bold and blackboard symbols.
\newcommand{\one}{\mathbf{1}}
\newcommand{\zero}{\mathbf{0}}
\newcommand{\I}{\mathbb{I}}  % For an indicator random variable.

% Common fractions.
\newcommand{\recip}[1]{\frac{1}{#1}}
\newcommand{\half}{\recip{2}}
\newcommand{\invsqrt}[1]{\recip{\sqrt{#1}}}

% Common summations.
\newcommand{\sumin}{\sum_{i=1}^n}
\newcommand{\sumtn}{\sum_{t=1}^n}

% Primitives.
\DeclareMathOperator{\clip}{clip}
\DeclareMathOperator{\vol}{vol}
\DeclareMathOperator{\sign}{sign}
\DeclareMathOperator{\suchthat}{such\ that}
\DeclareMathOperator{\subjto}{subject\ to}
\DeclareMathOperator{\otherwise}{otherwise}

% Delimiters.
\DeclarePairedDelimiter{\abs}{\lvert}{\rvert}
\DeclarePairedDelimiter{\norm}{\lVert}{\rVert}
\DeclarePairedDelimiter{\inner}{\langle}{\rangle}
\DeclarePairedDelimiter{\ceil}{\lceil}{\rceil}
\DeclarePairedDelimiter{\floor}{\lfloor}{\rfloor}

% Linear algebra and matrices.
\DeclareMathOperator{\rank}{\textbf{rank}}
\DeclareMathOperator{\Tr}{\textbf{Tr}}
\DeclareMathOperator{\diag}{diag}
% For sets of symmetric matrices
\renewcommand{\S}{\mathbb{S}}
\newcommand{\bmat}[1]{\begin{bmatrix} #1 \end{bmatrix}}

% Calculus.
\newcommand{\partialby}[1]{\frac{\partial}{\partial #1}}
% d for the measure / infinitesimal of an integral.
\renewcommand{\d}{\mathrm{d}}

% Optimization.
\DeclareMathOperator*{\argmax}{argmax\ }
\DeclareMathOperator*{\argmin}{argmin\ }
\DeclareMathOperator*{\minimize}{minimize\ }
\DeclareMathOperator*{\maximize}{maximize\ }

% Geometry.
\DeclareMathOperator{\conv}{conv}

% Probability distributions.
\DeclareMathOperator{\Normal}{\mathcal{N}}
\DeclareMathOperator{\Binomial}{Binomial}
\DeclareMathOperator{\Uniform}{Uniform}

% Other probability stuff.
\newcommand{\E}{\mathbb{E}\, }
\DeclareMathOperator{\iid}{iid}
\DeclareMathOperator{\Var}{Var}
\newcommand{\Indicator}[1]{\I_{\{#1\}}}
\newcommand{\cond}{\;\middle\vert\;}

% Learning stuff.
\DeclareMathOperator{\VCdim}{VCdim}
\DeclareMathOperator{\Ldim}{Ldim}
\DeclareMathOperator{\Reg}{Reg}
\DeclareMathOperator{\fat}{fat}
\DeclareMathOperator{\sfat}{sfat}
\DeclareMathOperator{\UCB}{UCB}

% Miscellaneous.
\renewcommand{\th}{^{\mathrm{th}}}
